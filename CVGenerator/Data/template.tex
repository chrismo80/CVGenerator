\newcommand{\CompanyName}{@Company}
\newcommand{\Address}{@Address}
\newcommand{\City}{@City}
\newcommand{\Recruiter}{@Recruiter}
\newcommand{\NewRole}{@NewRole}
\newcommand{\Product}{@Product}

\newcommand{\Name}{@Name}
\newcommand{\Location}{@Location}
\newcommand{\Mail}{@Email}
\newcommand{\Phone}{@Phone}
\newcommand{\Link}{@Link}
\newcommand{\Role}{@Role}
\newcommand{\Degree}{@Degree}
\newcommand{\Sign}{signature.png}
\newcommand{\Picture}{@Foto}


\newcommand{\Letter}{
Sie suchen einen leidenschaftlichen C\#-Entwickler? Mit über 15 Jahren Erfahrung in der Softwareentwicklung ist
diese Arbeit für mich weit mehr als nur ein Beruf – es ist eine Aufgabe, die ich mit Engagement und Freude angehe.
\newline

In dieser Zeit habe ich zahlreiche individuelle Desktop-Anwendungen entwickelt, von der Konzeption bis zur
Umsetzung. Dabei umfassten meine Aufgaben sowohl die eigenständige Umsetzung als auch die Unterstützung von
Kollegen in deren Projekten. Darüber hinaus entwarf und pflegte ich modulare Softwarekomponenten, die in
unterschiedlichen Projekten wiederverwendet werden konnten und so maßgeblich zur Verkürzung der
Entwicklungszeiten beitrugen.
\newline

Trotz meiner Erfahrungen mit verschiedenen Plattformen und Technologien blieb C\# stets der Kern meiner Arbeit.
Die Bereitstellung und Anbindung externer APIs, die Integration von SQL-Datenbanken mittels Entity Framework
sowie die Umsetzung von Datenvisualisierungen mit Microsoft Reporting Services gehörten stets zu den
wesentlichen Aspekten der Applikationen. Ich bin zuversichtlich, dass ich mich rasch in Ihre spezifischen Werkzeuge
und Technologien einarbeiten werde und somit zeitnah einen wertvollen Beitrag leisten kann.
\newline

Meine langjährige Tätigkeit im Sondermaschinenbau hat meine Fähigkeit geschärft, flexibel und lösungsorientiert
auf neue Anforderungen zu reagieren. Deshalb lege ich besonderen Wert auf die Lesbarkeit, Skalierbarkeit und
Wiederverwendbarkeit von Code, um nachhaltige Lösungen zu schaffen.
\newline

Nun suche ich ein neues berufliches Zuhause, das Softwareentwicklung als zentralen Bestandteil seiner Produkte
versteht – genau das reizt mich an der \CompanyName.
Lassen Sie uns doch in einem persönlichen Gespräch herausfinden, ob es in \City\ eine passende Position gibt,
beispielsweise im Bereich \Product, die ich für Ihr Unternehmen besetzen kann, um Ihr Team mit meiner
Expertise zu bereichern.
}


\newcommand{\Postskriptum}{
P.S.:
Um Ihnen einen ersten Eindruck meiner Code-Qualität zu geben, finden Sie ein \href{https://github.com/max/project}{\textbf{\ul{Repository}}} verlinkt,
in dem ich Avalonia als UI-Framework für Desktop-Anwendungen evaluiert habe.
}


\newcommand{\Timeline}{
    \begin{cvtimeline}{@MinYear}{@MaxYear}{21}{\linewidth}{Bildungsweg}{Berufserfahrung}
    @Education
    @Project
    @WorkExperience
    \end{cvtimeline}
}


\newcommand{\Summary}{
C\# Softwareentwickler mit über 15 Jahren Erfahrung im Bereich Automationssoftware im Sondermaschinenbau.
\newline

Kommunikationsstärken: Übersetzen von Anforderungen für Entwickler als auch Abstraktion von Softwarefunktionen für Projektmanager und Kunden.
\newline

Ausgeprägte Neugier und logisches Denkvermögen, problemlose Einarbeitung in neue Themengebiete, sowohl bei Werkzeugen als auch Technologien.
}


\newcommand{\Skills}{
    \Large{\textcolor{textcolor}{\textbf{Skills}}}
    \newline

    \small
    \mbox{
        \hspace{-10pt}
        \begin{barchart}{10}{1.75}{backcolor}{textcolor}{backcolor}{color1}{color2}{color3}
            \baritem{100}{Kommunikation}{100}{0}{0}
            \baritem{100}{Teamfähigkeit}{100}{0}{0}
            \baritem{100}{Versionsverwaltung}{90}{0}{0}
            \baritem{100}{Microsoft Reporting}{0}{80}{0}
            \baritem{100}{Entity Framework}{0}{70}{0}
            \baritem{100}{REST / gRPC}{0}{0}{20}
            \baritem{100}{XAML (Avalonia)}{0}{0}{20}
        \end{barchart}
        \hspace{10pt}
    }
}


\newcommand{\DiagramOne}{
    \Large{\textcolor{textcolor}{\textbf{Sprachen}}}
    \newline
    \begin{piechart}{360}{1.25}{backcolor}{textcolor}{backcolor}{\faKeyboard}
        \slice{70}{C\#}{color1}
        \slice{25}{SQL}{color2}
        \slice{5}{Rust}{color3}
    \end{piechart}
}


\newcommand{\DiagramTwo}{
    \Large{\textcolor{textcolor}{\textbf{Versionierung}}}
    \newline
    \begin{piechart}{360}{1}{backcolor}{textcolor}{backcolor}{\faCodeBranch}
        \slice{90}{Mercurial}{color1}
        \slice{10}{Git}{color3}
    \end{piechart}
}